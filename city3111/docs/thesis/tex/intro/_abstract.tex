% abstract
\cleardoublepage
\phantomsection
\addcontentsline{toc}
    {chapter}
    {Abstract}

\chapter*{Abstract}
Beowulf clusters offer great computational value for money as they can process a much greater amount of data than any single node can alone \cite{sterling_1995}. Free open-source (FOSS) implementations of various node clustering APIs now exist in abundance which make the creation of a high-performance computer (HPC) at home feasible. I intend on using one of these FOSS APIs to achieve a Beowulf cluster of my own using off-the-shelf hardware available to me. I undertake this project with a view to gain understanding of the current state in which non-commercial HPC is in, knowing that in future I might perform work using similar technology to further academic research fields such as bio-medicine, physics or engineering (amongst others).

Using my literature review, I concluded that an Open MPI cluster of five virtual nodes was able to outperform a single node by a magnitude close to the overall node increase. These results clearly show that using a larger pool of systems inter-connected by a TCP/IP network is able to offer a great speedup in computational power, even in spite of the relatively inexpensive nature of the underlying hardware, a trait that could be useful in research areas where budget is not of any particular depth.
